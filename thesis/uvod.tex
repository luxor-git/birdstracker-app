\chapter*{Úvod}
\addcontentsline{toc}{chapter}{Úvod}

S rozvojem IoT technologí dochází ke stále další a další miniaturizaci autonomních off-the-grid zařízení. Jedním z mnoha oborů, které z tohoto vývoje těží, je zoologie. Obory zoologie zabývající se výzkumem migrací, studiem životních cyklů, ochrany a výzkum vlivu lidské činnosti na zvířata díky tomuto vývoji využívají stále dostupnější trackovací zařízení nasazovaná na zvířata. Nasazená zařízení komunikují především pomocí GSM technologií a sbírají telemetrická data o životních funkcích zvířete, pozici, vzdálenosti k zájmovým bodům a podobně.

% proč bylo téma zvoleno	

Cílem této práce je vytvoření aplikace pro sledování a kontrolu divokých zvířat v terénu, konkrétněji ptáků. Primární funkcí aplikace je zobrazení posledních pozic vybraných zvířat v mapě. O vybraných zvířatech se ukládají metainformace, které mohou sloužit k dodatečné identifikaci v terénu. Pro aplikaci je kritická možnost fungování bez internetového připojení, kterou současné řešení nepodporuje. Současné řešení taktéž není vhodné pro použtí na mobilu z hlediska použitelnosti. Toto téma je řešeno z důvodu absence efektivního řešení tohoto problému.

Zvolený způsob řešení je multiplatformní aplikace vyvíjená v prostředí React Native s možností prací offline. React Native umožňuje vytváření mobilních aplikací pro platformy Android i iOS bez nutnosti psát dvě separátní aplikace. React Native je souborem JS knihoven postavených nad frontendovým frameworkem React od společnosti Facebook. Pro zrychlení vývoje bez nutnosti podbroného testování aplikací na obou platformách byla použita nástavba Expo, která dále abstrahuje od platformně závislého kódu. Zdrojem dat je ornitologická platforma Anitra, která mimo funkce datového uložiště podporuje sdílení dat a udržování metainformací o žvířatech.

Popsat kapitoly

Metodika??