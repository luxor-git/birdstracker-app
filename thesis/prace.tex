%% Verze pro jednostranný tisk:
%\documentclass[11pt,a4paper]{report}
%\usepackage[top=25mm,bottom=25mm,right=25mm,left=30mm,head=12.5mm,foot=12.5mm]{geometry}
%\let\openright=\clearpage

%% Pokud tiskneme oboustranně:
\documentclass[11pt,a4paper,twoside,openright]{report}
\usepackage[top=25mm,bottom=25mm,right=25mm,left=30mm,head=12.5mm,foot=12.5mm]{geometry}
\let\openright=\cleardoublepage

%% Definice různých užitečných maker (viz popis uvnitř souboru)
\input{makra}

%%% Údaje o práci
% Název práce v jazyce práce (přesně podle zadání)
\def\NazevPrace{Vývoj aplikace pro sledování zvířat}

% Typ práce
\def\TypPrace{BAKALÁŘSKÁ PRÁCE}
%\def\TypPrace{DIPLOMOVÁ PRÁCE}


% Jméno autora
\def\AutorPrace{Karel Douda}

% Rok odevzdání. měsíc (slovně)
\def\DatumOdevzdani{květen 2020}

% Vedoucí práce: Jméno a příjmení s~tituly
\def\Vedouci{Ing. David Král}

% Studijní program a obor
\def\StudijniProgram{Aplikovaná informatika}
\def\StudijniObor{Aplikovaná informatika}

% Text čestného prohlášení pro MUŽE pro bakalářskou prácí
\def\Prohlaseni{Prohlašuji, že jsem bakalářskou práci \textit{\NazevPrace} vypracoval samostatně za použití v práci uvedených pramenů a literatury.}

\def\Podekovani{%
Poděkování.
}

% Abstrakt (doporučený rozsah cca 150-250 slov; nejedná se o zadání práce)
\def\Abstrakt{%
Tato bakalářská práce se zabývá vývojem mobilní aplikace pro zjednodušení práce ornitologů v poli, konkrétněji o zobrazení dat z GPS-GSM trackerů, zobrazení informací o zvířatech co je nosí a jejich atributy, anotace a přiložená média. Vytvořená aplikace komunikuje s ornitologickou platformou Anitra a slouží jako její klient. Hlavním přínosem práce možnost využít aplikaci na mobilních zařízeních a místech se špatným pokrytím bezdrátového internetu. Cílem práce je vytvořit mobilní offline-capable aplikaci vyhovující požadavkům ornitologů.

V práci je ve zkratce popsán současný stav ve světě zvířecí telemetrie, specifikace problémových oblastí pro práci v poli, popis současných řešení určitých částí problému a řešení jsou na konci srovnána. Práce je rozdělena do šesti kapitol, kde první dvě kapitoly se věnují již zmíněným vývojům v oblasti telemetrie a popisu současné situace řešení vytipovaných problémů.

Po stanovení problémů dochází ke krátké analýze problémů. V následující kapitole se vybírá technologie dle kritérií stanovených pro aplikaci tohoto typu. Navazující kapitola se věnuje návrhu uživatelského rozhraní a návrhu aplikační architektury. Poslední kapitola se věnuje implementaci aplikace a jde do detailu pro některé části aplikace, které jsou řešeny zajímavým způsobem či jsou považovány za reprezentativní pro ostatní části aplikace. Aplikace nemohla být nasazena a otestována zákazníky kvůli situaci spojené s COVID-19, které způsobily zdržení v procesu publikace mobilních aplikací.

Výsledkem této práce je funkční, ačkoliv v momentu vydání práce nevydaná aplikace pro práci ornitologů v poli.
}
\def\AbstraktEN{%
The purpose of this thesis is to document the development of a mobile application for fieldwork support of ornithologists, specifically the display of data from GPS-GSM trackers, their associated animals and the attributes, annotations and media related to those animals. The finished application communicates with the ornithological web platform Anitra and serves as its client. Main benefit of this thesis is the creation of a mobile application for fieldwork support that functions well in areas with poor wireless internet coverage. The aim of this thesis is to develop a mobile, off-line capable application that meets the demands of ornithologists.

The text itself contains the summarization of development and current state of animal telemetry, specification of problem areas for ornithological field work and in the end, presents current methods of solving those problems and their pros and cons. This thesis is is separated into six chapters, where the first two chapters are concerned with the previously described developments in animal telemetry and problem areas identified in the first chapter.

Problem area specification is followed by requirement analysis. Following chapter describes the process of technology stack selection. Next chapter is concerned with software and UI design. Finally, the last chapter describes the process of implementing the application, with special emphasis on unique or representative segments of the application. Unfortunately the application could not be deployed or tested by end users in time due to the COVID-19 situation which caused complications in the application deployment process.

Nevertheless, the result of this thesis is a working application created to suit the needs of the ornithologists which, at the moment of writing, is not yet published on any mobile application store.
}

% 3 až 5 klíčových slov (doporučeno)
\def\KlicovaSlova{animal tracking, mobilní aplikace, react native, expo}
\def\KlicovaSlovaEN{animal tracking, mobile app, react native, expo}

%% Titulní strana a různé povinné informační strany
\begin{document}
\include{zacatek}

%%% Strana s automaticky generovaným obsahem bakalářské práce
\setcounter{tocdepth}{2}
\tableofcontents

%%% Obrázky v bakalářské práci
\openright
\listoffigures

%%% Tabulky v bakalářské práci (opět nemusí být nutné uvádět)
\clearpage
\listoftables

%%% Seznam procedur
\lstlistoflistings

%%% Použité zkratky v bakalářské práci (opět nemusí být nutné uvádět)
\chapter*{Seznam použitých zkratek}

\begin{multicols}{2}
\raggedright
\begin{description}
\item [IoT] Internet of Things
\item [GSM] Global System for Mobile Communications
\item [GPS] Global Positioning System
\item [HTTP] HyperText Transfer Protocol
\item[CSS] Cascading Style Sheets
\item [JS] JavaScript
\item [SDK] Software Development Kit
\item [NPM] Node Package Manager
\item [ES] EcmaScript
\item [TS] TypeScript
\item [RN] ReactNative
\item [JSON] JavaScript Object Notation
\end{description}
\end{multicols}



\pagestyle{fancy}
%%% Jednotlivé kapitoly práce jsou pro přehlednost uloženy v samostatných souborech
\chapter*{Úvod}
\addcontentsline{toc}{chapter}{Úvod}

S rozvojem IoT technologí dochází ke stále další a další miniaturizaci autonomních off-the-grid zařízení. Jedním z mnoha oborů, které z tohoto vývoje těží, je zoologie. Obory zoologie zabývající se výzkumem migrací, studiem životních cyklů, ochrany a výzkum vlivu lidské činnosti na zvířata díky tomuto vývoji využívají stále dostupnější trackovací zařízení nasazovaná na zvířata. Nasazená zařízení komunikují především pomocí GSM technologií a sbírají telemetrická data o životních funkcích zvířete, pozici, vzdálenosti k zájmovým bodům a podobně.

% proč bylo téma zvoleno	

Cílem této práce je vytvoření aplikace pro sledování a kontrolu divokých zvířat v terénu, konkrétněji ptáků. Primární funkcí aplikace je zobrazení posledních pozic vybraných zvířat v mapě. O vybraných zvířatech se ukládají metainformace, které mohou sloužit k dodatečné identifikaci v terénu. Pro aplikaci je kritická možnost fungování bez internetového připojení, kterou současné řešení nepodporuje. Současné řešení taktéž není vhodné pro použtí na mobilu z hlediska použitelnosti. Toto téma je řešeno z důvodu absence efektivního řešení tohoto problému.

Zvolený způsob řešení je multiplatformní aplikace vyvíjená v prostředí React Native s možností prací offline. React Native umožňuje vytváření mobilních aplikací pro platformy Android i iOS bez nutnosti psát dvě separátní aplikace. React Native je souborem JS knihoven postavených nad frontendovým frameworkem React od společnosti Facebook. Pro zrychlení vývoje bez nutnosti podbroného testování aplikací na obou platformách byla použita nástavba Expo, která dále abstrahuje od platformně závislého kódu. Zdrojem dat je ornitologická platforma Anitra, která mimo funkce datového uložiště podporuje sdílení dat a udržování metainformací o žvířatech.

Popsat kapitoly

Metodika??
%%% Fiktivní kapitola s ukázkami sazby

\chapter{Problémová oblast}{\tiny }

Sledování a popis životního cyklu ptáků, vědecká disciplína nazývaná ornitologie, je poměrně starou vědeckou disciplínou s kořeny již ve starověku. Jedním z prvních dochovaných textů zabývajícím se popisu života ptáků je Aristotelova \emph{Historia Animalium} v roce 350 př. n. l \cite{historiaAnimalium}, která podrobně popisovala mimojiné i ptačí migrace. Ornitologie se postupně rozvíjela -- zaváděla se taxonomie ptactva, studovala se ptačí anatomie, ale spousta otázek spojených s chováním ptáků zůstavala nezodpovězena. V roce 1805 se Americký ornitolog John James Audubon údajně snažil prokázat, že se každým rokem na jeho farmu vrací stejný jedinec druhu \emph{Sayornis phoebe} přivázáním stříbrného lanka na nohu \cite{halley2018audubon}. Tímto by nepřímo položil základy pro tzv. kroužkování, pasivní označení jedinců kovovým kroužkem s vyraženým sériovým identifikátorem kroužku. Systém kroužkování zavedl dánský ornitolog a učitel Hans Christian Cornelius Mortensen v roku 1899. Na území České republiky se kroužkování ujalo roku 1910, přibližně rok po rozšíření systému v Anglii a Německu. Dle informací Společnosti spolupracovníků kroužkovací stanice Národního muzea je dnes registrováno 480 spolupracovníků, kteří ročně okroužkují kolem 175 000 ptáků \cite{krouzkovaniPtakuHistorie}.

Kroužkování ptáků slouží k mnoha účelům -- mapování migračních tras (zimoviště, návrat na stejné lokace), populační studie (např. rozšiřování jednotlivých druhů, roční úhyn nebo naopak nárůst), životní cyklus ptactva (např. délka života). Kroužkování je dodnes základem práce ornitologů díky jeho nízké cenové náročnosti a mezinárodní kolaboraci ornitologů, záchranných stanic a dobrovolníků \cite{krouzkovaniPtaku}. 

Hlavní nevýhodou kroužkování je samozřejmě pasivní povaha této metody -- o ptácích se zjišťují pouze kusé informace o jejich přibližné poloze, které lze získat pouze odchytem jedince, kroužky samotné nelze vzdáleně odečítat. Metoda je také účinná jen při velkém počtu kroužků \cite{sokolov2011modern}. Za vylepšení umožňující i vzdálený odečet lze považovat barevné kroužky, křídelní značky či jiné barevně odlišitelné a dostatečně velké identifikační prvky umístěné na zvířeti. Barevné kombinace jednoznačně určují výzkumný projekt, v rámci kterého bylo zvíře označeno a konkrétního jedince. 

Kroužkování závisí také na vysoké angažovanosti dobrovolníků a disciplinované administrativě související s osazením a nahlášením pozorování ptáků s kroužky či s barevným označením. Dále taktéž neřeší jiné úkoly ornitologů, které souvisí např. s kontrolou hnízd -- musí se zkontrolovat rozsáhlý počet hnízďištních lokalit místo konkrétních hnízd, což s sebou přináší logistické i administrativní problémy.
% Pozorování samotné je taktéž záležitostí s prvkem nejistoty - kroužky nemusí být dostatečně čitelné pro kompletní identifikaci, pozorovatel tedy může informovat např. jen o pozorování určitého druhu ptáka. 

\section{Vývoj v oblasti telemetrie}

Počátek radiotelemetrie životních ukazatelů zvířat je v čase obtížné zařadit, ale do určitého bodu v čase se jednalo téměř výhradně o invazivní procedury omezující pohyb, což značně znehodnocovalo získané výsledky. Jeden z prvních úspěšných neinvazivních experimentů vyústil ve vynález radiového induktografu, zařízení měřící fyziologickou aktivitu zvířete bez omezování pohybu \cite{fuller1948radio}.

Miniaturizací elektronických součástek, především akumulátorů, se v 60. letech 20. století začínají rozšiřovat aktivní telemetrická zařízení. Klíčovým pro tento rozvoj byl vynález transistoru v roce 1948, resp. jeho obecná dostupnost v roce 1952, což umožnilo nástup kompaktní radiotelemetrie \cite{amlaner2013handbook}. Radiolokátory mohou pomocí elektromagnetických pulsů přenášet určitá data za použití modulací nosné vlny signálu. Jednoduché radiolokátory umožnily v terénu přesně určit pozici zvířete sledováním intenzity signálu triangulací \cite{Farve2014}, měření signálů z tří různých lokalit a následné aproximaci polohy z těchto měření. Tímto se zjednodušilo např. hledání hnízd, čímž se taktéž usnadnilo hledání a označení mladých jedinců, kteří ještě nebyli vyvedeni z hnízda. Pokročilejší dataloggery umožnily sbírat data a následně pomocí elektromagnetických vln na vyžádání odeslat data příjemci, čímž např. oproti triangulaci signálu vysílače lze zjistit i historické informace, ale pro získání těchto informací byla stále nutná fyzická blízkost ke zvířeti.

% https://en.wikipedia.org/wiki/GPS_wildlife_tracking

Ornitologie těžila i z rozvoje kosmických programů. Signály z dostatečně výkonných radiolokátorů mohou být přijaty speciálními družicemi systému Argos v kosmu a za pomoci Dopplerova jevu lze spočítat přibližnou polohu daného jedince \cite{Farve2014}. Tato řešení nebyla zpočátku pro ornitologii příliš vhodná z hlediska hmotnosti radiolokátorů. Příchodem GSM sítí v 90. letech a uvolňování restrikcí na použití GPS se situace pro ornitology zásadně změnila. Na trhu se objevily výrazně lehčí (desítky až jednotky gramů) GPS-GSM trackery vhodné i pro malé druhy ptáků \cite{sokolov2011modern}. Data z těchto trackerů se průběžně sbírají a odesílají do systémů výrobců zařízení, případně přímo majiteli zařízení pomocí GSM technologií (konkrétněji SMS a mobilních dat).

Rapidní nástup webových technologií po roce 2000 umožnil výrobcům zařízení jednoduše uživatelům poskytovat navazující služby ke svým zařízením, jmenovitě jednoduchou konfiguraci zařízení, základní visualizaci dat, exporty dat, vedení metainformací k zařízení (např. na jakém zvířeti zařízení je) a další. Většina systémů se omezuje pouze na trackery od jednoho výrobce a funkce mimo množinu konfigurace zařízení jsou primitivní, až nedostatečné. I přesto se tyto systémy osvědčily např. v hledání otrávených jedinců za použití analytických funkcí těchto aplikací přímým i nepřímým způsobem \cite{stoynov2018early}.

V roce 2017 vznikla česká ornitologická platforma Anitra, která poskytuje komplexní nástroje pro správu zařízení a visualizaci dat od širokého spektra výrobců trackerů \cite{krouzkovaniPtakuAnitra}. Platforma Anitra taktéž poskytuje správu metainformací o zvířatech, správu zájmových bodů, nahrávání příloh a komplexní metody sdílení dat. Pro tuto práci byla platforma Anitra vybraná z důvodu autorovy možnosti vytvářet API na míru mobilní aplikaci a již existující uživatelské základny. Anitra taktéž prodává a vyvíjí vlastní GPS-GSM zařízení.

Pro základní představu typického systému GPS-GSM trackerů (konkretizován pro platformu Anitra) je níže uvedeno schématické znázornění jeho prvků a vztahů mezi nimi.

\begin{figure}[h]
	\includegraphics[width=\linewidth]{img/diagram_system.png}
	\caption{Schéma systému GPS-GSM trackerů}
	\label{fig:boat1}
\end{figure}
%%% Fiktivní kapitola s ukázkami sazby

\chapter{Existující řešení}

% popsat, jak se současně využívá Google MyMaps, fyzický zápisník, (Anitra?), Movebank aplikace
%%% Fiktivní kapitola s ukázkami sazby

\chapter{Analýza požadavků}

Kapitola analýza požadavků se věnuje první fázi ve vývoji softwarového projektu. Analýza požadavků se skládá z různých fází dle použité metodiky, ale vždy je cílem od relevantních stran získat seznam požadavků, co aplikace musí splňovat, aby naplnila cíle, kvůli kterým se k vývoji SW projektu rozhodlo. V jiných slovech je tedy část analýzy požadavků kritická pro úspěch softwarového projektu \cite{maguire2002user}.

% chce to zdroje, a ten text je taky dost divný

\section{Identifikace stakeholderů}

% vysvětlit pojem stakeholder, jak se dělá analýza stakeholderů

Identifikace stakeholderů je první fází v analýze požadavků. Účelem této fáze je identifikovat všechny relevantní zúčastněné strany, aby mohly být zapojeny do fáze sběru požadavků a přiřadit k nim jednotlivé případy užití aplikace \cite{maguire2002user}. V případě nedostatečného zapojení stakeholderů je vysoká pravděpodobnost, že software nebude uspokojovat potřeby všech zúčastněných stran.

Za stakeholdery byly identifikovány následující subjekty: stávající uživatele webové aplikace Anitra a majitel firmy Anitra System s.r.o.

Stávající uživatelé aplikace již požadovali funkce pro práci v terénu a jejich požadavky byly zaevidovány do pořadníku funkcí pro webovou aplikaci. Tyto požadavky byly konzultovány se zakladatelem firmy Anitra a byly vybrány k řešení. Zakladatel firmy je projektovým manažerem a komunikuje se zákazníky na denní bázi, má tedy dostatečný přehled o požadavcích zákazníků a jejich priorit, zároveň působí i jako kontaktní bod podpory pro aplikaci. Cílem zakladatele firmy je poskytnout uživatelům nástroje pro efektivní práci v terénu i pro rychlou kontrolu dat z mobilních zařízení, což by mohlo být bráno jako konkurenční výhoda, jelikož na trhu nebyla nalezena podobná aplikace.

\section{Sběr požadavků}

Pro zakladatele firmy Anitra je cílem mobilní aplikace doplnění ekosystému značky Anitra o vhodné řešení zobrazení a zadávání dat v terénu. Na začátku práce byla pouze dostupná webová aplikace, která měla pro práci v terénu následující nevýhody:

\begin{itemize}
	\item data po odpojení z internetu nebyla dostupná, ve většině případů ani již načtená data,
	\item aplikace byla příliš náročná na energii,
	\item mapové podklady se nezobrazovaly po odpojení ze sítě,
	\item aplikace byla na datový přenos příliš náročná,
	\item navigace v GUI byla pro malá zařízení příliš složitá.
\end{itemize}

Tyto podněty byly sebrány od uživatelů webové aplikace Anitra v období 2018-2020, ale mělo by se jednat o dostatečný základ pro zařazení mobilní aplikace do portfolia ekosystému Anitra, jelikož se jedná o obecné předpoklady pro software tohoto typu.

Z těchto zkušeností uživatelů vyplynuly klíčové funkční i nefunkční požadavky na funkcionalitu aplikace. Jednotlivé části jsou rozebrány níže. Další funkční požadavky byly definovány zakladatelem firmy Anitra (zde uvedeny ve zkrácené formě, seřazeny dle priority):

\begin{itemize}
	\item zobrazit poslední polohu zařízení v mapě,
	\item možnost stáhnout si mapy před kontrolou na místě,
	\item zobrazení dat ze zařízení,
	\item zobrazení vlastní pozice na mapě
	\item notifikace o nových datech,
	\item zaznamenat trasu, bod,
	\item možnost přiložit obrázek k zvířeti.
\end{itemize}

\section{Funkční požadavky}

Funkční požadavky popisují očekávané chování systému, ale nevěnují se technickým detailům, jak má systém fungovat. Funkční požadavky slouží pro splnění potřeby uživatelů a lze je vyjádřit ve formě případů užití nebo diagramu případu užití \cite{jacobson1987object}.

Jednotlivé případy užití jsou rozebrány v kapitole uvedené níže.

\subsection{Diagram případů užití}

Diagram případu užití je diagramem ze standardní sady UML využívané v softwarovém inženýrství.

\begin{figure}[H]
	\begin{center}
		\includegraphics[width=140mm]{img/usecase.png}
	\end{center}
	\caption[Diagram případu užití]{Diagram případu užití -- zdroj: autor}
\end{figure}

Diagram případů užití byl sestaven dle nasbíraných požadavků zmíněných v minulé kapitole. Případy užití byly rozšířeny o nevyslovené, ale odvoditelné požadavky, např. možnost přihlásit se a odhlásit se.

\section{Nefunkční požadavky}

Nefunkční požadavky udávají kvalitu systému, ale ne jeho chování. Tyto požadavky často přináší omezení do implementací funkčních požadavků a je někdy obtížné tyto požadavky sladit, mohou působit i protichůdně. Jako příklad se často uvádí požadavky spojené s komfortem uživatelů (např. přístupnost, výkonnost, lokalizace), provozem aplikace (přenositelnost, tolerance chyb) a bezpečnost. Do nefunkčních požadavků se řadí i požadavky související s možností dalšího vývoje systému \cite{chung2012non}.

Z výše uvedeného seznamu požadavků vyplynulo několik klíčových nefunkčních požadavků především ve vztahu k uživatelům. Pro úspěch aplikace je nutné, aby grafické rozhraní aplikace bylo dostatečně jednoduché a podstatné funkce byly rychle přístupné bez složitých navigací, jelikož uživatel nemusí mít mnoho času pro práci s aplikací v terénu. S tímto souvisí i nutnost odezvy v jednotkách sekund. 

Neuvedeným požadavkem je multiplatformnost aplikace, jelikož část uživatelů stávající aplikace využívá mobilní operační systém iOS a část Android. Podstatným požadavkem je i rozšiřitelnost aplikace, jelikož se s vývojem aplikace počítá i po dokončení této práce.

\section{Časové aspekty}

Aplikace má být vyvinuta a nasazena nejpozději do konce května, ideálně již v dubnu. V tuto dobu probíhá nejvyšší počet ornitologických operací spojených s nasazováním trackerů, kontrolou hnízd a dalších. Aplikaci nebylo možné začít vyvíjet dříve než v polovině února kvůli návaznosti na API již existující webové aplikace. Pro prvotní verzi aplikace je kritické aby spolehlivě zobrazovala poslední body v mapě a po prvotní synchronizaci fungovala bez internetového připojení.

%%% Fiktivní kapitola s ukázkami sazby

\chapter{Technologie}

\section{TypeScript a EcmaScript}

\section{React}

\section{React Native}

\section{Expo}

\section{TurtleCLI}

TurtleCLI je nástroj pro sestavení Expo aplikací z příkazové řádky. Nástroj je určen pro uživatele, kteří si nepřejí využívat infrastruktury poskytované od Expo pro sestavování vlastních aplikací, ale přejí si aplikaci sestavit například na vlastním stroji nebo na vlastním nástroji kontinuální integrace. Autor práce vybral tuto technologi z důvodu vysoké vytíženosti veřejné infrastruktury společnosti Expo (varianta nazývaná Community), kde jeho build ve frontě čekal v jednotkách hodin, než se dostal na řadu. Společnost Expo nabízí placenou variantu svých služeb, ve kterých avizuje výrazně vyšší rychlost odbavení, než u varianty zdarma a nástroje pro týmovou práci. Za značnou nevýhodu se taktéž může brát fakt, že kód je vždy odesílán třetí straně k sestavení, což pro mnoho firem může být nepřípustné. Oproti tomu provozování na vlastní infrastruktuře skýtá mnoho potenciálních výhod, za zmínku stojí napojení na firemní CI procesy, testovací infrastrukturu, rapidní nasazení mnoha paralelních větví do testovacích kanálů, jednodušší napojení sestavení na jiné akce. Níže je uvedena tabulka popisující podstatné rozdíly mezi jednotlivými způsoby sestavení Expo aplikací. 

\begin{table}[h]
	\begin{tabular}{llll}
		Kategorie                        & Community         & Priority      & Vlastní infrastruktura                 \\
		Cena                             & zdarma            & 29 USD / měs. & cena výpočetního výkonu                \\
		Čekací doba na začátek sestavení & cca. 1h           & minuty        & ihned                                  \\
		Složitost zprovoznění            & součástí expo-cli & triviální     & dle zvolené varianty, nepříliš náročné \\
		Provázanost s Expo               & vždy              & vždy          & dle zvolené varianty                  
	\end{tabular}
\end{table}
%%% Fiktivní kapitola s ukázkami sazby

\chapter{Návrh}

Kapitola návrh se věnuje dvou tématům. V první části se věnuje návrhu uživatelského rozhraní a přechodů mezi obrazovkami. V druhé části se věnuje tradičnějšímu chápání slova návrh ve vývoji, a to softwarové architektuře. Cílem návrhu je vytvořit 

\section{Uživatelské rozhraní}

Návrh uživatelského rozhraní je disciplínou softwarového inženýrství, která se věnuje návrhu uživatelského rozhraní na koncepční úrovni. Při vytváření uživatelského rozhraní je nutno brát v potaz požadavky uživatelů, aby nebylo ubíráno na efektivitě aktivity uživatele v aplikaci. Cílem je navrhnout uživatelské rozhraní pro uživatele rychlo uchopitelné a přehledné. V disciplíně návrhu uživatelského rozhraní se využívá mnoho modelovacích technik, např. prototypování a drátěné modely. Pro správné posouzení efektivnosti uživatelského rozhraní je nutno tyto modely konfrontovat s potenciálními uživateli a sledovat jejich reakce a přpomínky a reflektovat je v následujících iteracích návrhu uživatelského rozhraní. 

Pro tuto práci byla zvolena technika drátěných modelů (tzv. wireframe), schematickým nákresem prvků na obrazovce a jakým způsobem do sebe zapadají (ref). Tyto návrhy jsou ušetřeny komentářů, které součástí wireframů někdy bývají. V rámci této práce nebylo možno provádět studie mezi potenciálními uživateli, byl zvolen alternativní způsob inspirace známými UI elementy, které jsou přítomny v jiných aplikacích. 

% https://www.academia.edu/6511543/The_Elements_of_User_Experience_User-Centered_Design_for_the_Web_and_Beyond_Second_Edition p. 128

\subsection{Onboarding, přihlášení, registrace}

\begin{figure}[h]
	\begin{center}
		\includegraphics[width=70mm]{img/wf_onboarding.png}
		\includegraphics[width=70mm]{img/wf_login.png}
	\end{center}
	\caption[Wireframe přihlášení a onboarding obrazovky]{Wireframe onboarding a přihlašovací obrazovky -- zdroj: autor}
	\label{fig:boat1}
\end{figure}

Pro přihlašovací obrazovku byl zvolen koncept tzv. onboardingu, praktiky popisující funkce aplikace a výhody založení účtu. Onboarding lze také chápat jako náhradu aplikačního návodu. Onboarding může být prováděn mnoha způsoby, jako nejsnadněji řešitelný byl vybrán tzv. carousel, seznam posouvatelných panelů, kde každý z nich obsahuje určité informace o mobilní aplikaci. Pro přihlašovací obrazku byl zvolen očekávatelný klasický formát přihlašovacího dialogu. Pro registraci byl využit tzv. web view, komponenta, co zobrazí webový prohlížeč svázaný s aplikací. Registrace se tedy provádí přes stávající webovou aplikaci. Důvodem tohoto rozhodnutí je absence API pro vytváření uživatelských účtu, ale zároveň i předpoklad, že uživatel mobilní aplikace bude i zároveň uživatelem aplikace webové.

\subsection{Mapová koncepce}

Hlavní obrazovkou aplikace byla zvolena obrazovka s mapou, aby odpovídala očekávání uživatelů webové aplikace Anitra, kde hlavní obrazovkou je taktéž mapa. Veškeré kontrolky, které by se daly zobrazit jako separátní stránky, se zobrazují v modálních oknech nad obrazovkou, což je opět koncepce vycházející z již hotové webové aplikace. Aplikace tedy rovnou splní jeden ze zmíněných případů užití při zapnutí, a to kontrolu posledních pozic trackerů a případné odfiltrování pro uživatele nezajímavých trackerů způsoben srovnatelným s aktuálním řešením, čímž by se dalo očekávat zkrácení doby učení uživatelů aplikace.

\section{Aplikační architektura}

%Clements, Paul; Felix Bachmann; Len Bass; David Garlan; James Ivers; Reed Little; Paulo Merson; Robert Nord; Judith Stafford (2010). Documenting Software Architectures: Views and Beyond, Second Edition. Boston: Addison-Wesley. ISBN 978-0-321-55268-6.

Aplikační (resp. softwarová) architektura popisuje strukturu softwarového systému. Struktura softwarového systému obsdahuje softwarové elementy, jejich vztahy a vlastnosti elementů a jejich vlastností (ref). Přístupy k této disciplíně se liší dle metodiky vývoje softwaru, rigozórní metody preferují architekturu dopředu pevně formulovat, zatímco více agilní metodiky pravý opak. V rámci této práce softwarová architektura nebyla řešena příliš do podrobna, jelikož se jedná o velmi malou aplikaci.

\subsection{Základní koncepce}

Pro řešení aplikace byly zvoleny architektonické vzory, tedy typické architektury, které se v návrhu softwarových architektur často vyskytují. Aplikace byla navrhnutá jako dvouvrstvá klient-server aplikace, kde první vrstva zajišťuje operaci s daty (modelová vrstva, podkapitola Datové zdroje a entity) a druhá vrstva správné zobrazení (prezentační vrstva, podkapitola Komponenty). Tento koncept byl zvolen díky plnému splnění architektonických nároků malé aplikace a komponentové architektuře React.

\subsection{Datové zdroje a entity}

\subsection{Komponenty}


%%% Fiktivní kapitola s ukázkami sazby

\chapter{Implementace}

Kapitola implementace se věnuje konkrétním akcím vedoucím k vytvoření spustitelného aplikačního artefaktu. V jednotlivých podkapitolách budou uvedeny využité prostředky, konkrétní kroky a jejich výstupy. Některé z těchto kroků jsou použitelné pro všechny Expo projekty a některé úsudky lze využít při vytváření vlastních Expo projektů. Za zmínku zde také stojí kapitola Implementace off-line map, ve které je popsán způsob jak zajistit funkční off-line mapy pro Expo projekty, pro co neexistují zatím žádné komunitní knihovny nebo Expo-nativní způsoby řešení.

\section{Vývojové prostředí}

% https://www.techopedia.com/definition/16376/development-environment

Vývojové prostředí je kolekce procedur a nástroj pro vývoj, testování a ladění aplikací nebo programů \cite{technopediaDevEnv}. Termín vývojové prostředí se používá v synonymu s termínem IDE, integrovaným vývojový prostředím, což je softwarový nástroj pro psaní, sestavení, testování a ladění programů. Do tohoto termínu taktéž lze zahrnout i nástroje co IDE samotné využívá pro setavení artefaktů.

Pro vývoj aplikace byly zvoleny následující nástroje vývojového prostředí:

\begin{itemize}
	\item VisualStudio Code jako nástroj IDE,
	\item Git jako verzovací nástroj,
	\item GitHub jako nástroj verzování i projektového řízení,
	\item AndroidStudio pro správu Android virtuálních strojů,
	\item XCode pro správu iOS virtuálních strojů (pouze na MacOS).
\end{itemize}

Dalším klíčovým softwarovým nástrojem je taktéž instalace Node.js a balíčkovacího nástroje NPM, bez kterých by byl vývoj Expo aplikace nemožný. Text této práce se jejich instalaci nevěnuje a předpokládá, že již nainstalované jsou. Instalace na běžných a aktuálních verzích OS není problémem.

Do pojmu vývojové prostředí se zahrnují i izolovaná prostředí, ve kterém běží testovací nebo produkční instance aplikace (resp. datového zdroje), v tomto případě platformy Anitra. Pro vývoj této aplikace nebylo toto rozlišení nutné a vyvíjelo se přímo na produkční verzi platformy Anitra.

\subsection{Vytvoření projektu}

V ukázce kódu \ref{expoinstall} bylo již zmíněno, jakým způsobem vytvořit Expo projekt. Pro vytvoření expo projektu stačí zadat následující příkazy do příkazové řádky.

\begin{lstlisting}[language=Bash, caption=Vytvoření nového projektu]
npm install --global expo-cli
expo init mobilni-aplikace
\end{lstlisting}

První řádek nainstaluje \emph{expo-cli}, nástroje pro vytváření, správu, spouštění a sestavení Expo projektů. Druhý příkaz vytvoří nový expo projekt \emph{mobilni-aplikace} v aktuální složce uživatele.

Z předpřipraveného kódu je dobré zmínit vstupní bod aplikace \emph{App.tsx}, ve kterém je připravená ukázková komponenta. Z tohoto vstupního bodu se implementují navigátory zmíněné v kapitole Implementace navigátorů.

\section{Implementace úložišť a entit}

Implementace úložišť a entit byla vytvořena dle návrhu aplikace. Bylo vytvořeno více entit, než bylo plánováno, aby celá aplikace komunikovala za pomocí typovaných objektů a tím se zjednodušil proces ladění v aplikaci i urychlil vývoj. Pro entity bylo využito rozhraní, aby entity pro funkce ukládání měly stejnou signaturu a bylo je možné používat ve stylu OOP bez složitých konstrukcí. V ukázce kódu uvedené níže je ukázáno rozhraní, které musí implementovat všechny entity a entity, které mají být uložitelné na disk.

\begin{lstlisting}[language=JavaScript, caption=Ukázka entity]
export interface IEntity {
	id?: number;
	
	synchronized: boolean;
	lastSynchronized?: Date;
};

export interface ISerializableEntity extends IEntity {
	toJson() : object;
	
	toJsonString() : string;
	
	fromJson(json: any): IEntity;
};
\end{lstlisting}

Implementace úložišť samotných spočívá v implementaci tří částí -- v komunikaci se serverem, transformaci dat od serveru a uložení těchto dat na disk. Pro komunikaci se serverem pomocí HTTP Rest API byla zvolena knihovna \emph{axios}, která se běžně používá v JavaScriptových projektech pro abstrakci rozdílů mezi jednotlivými klienty. Knihovna dále umožňuje jednodušší přístup např. k hlavičkám HTTP dotazu a je tedy jednoduché se např. oproti vzdálenému serveru autentifikovat. Axios očekává na vstupu objekt obsahující výčet parametrů a na výstupu vrací tělo dotazu zformátované do správného formátu (např. když server vrací JSON odpověď, Axios z vráceného textu správně složí JSON).

Transformace dat ze serveru spočívá v převodu klíčů a hodnot vrácených ze serveru do předem zmíněných entit. Entity v aplikaci neobsahují celá data, ale pouze výčet, který aplikace skutečně potřebuje. Tímto se šetří nejen místo na disku, ale i doba serializace a deserializace entit při čtení z disku. V API nejde stanovit, jaká pole dotazovatel vyžaduje a filtrování je tedy nutno provést až na koncovém bodu procesu, tedy v mobilní aplikaci.

Pro uložení dat byla využita součást Expo \emph{expo-file-system}, která abstrahuje od jednotlivých souborových systémů operačních systémů. Nevýhodou tohoto systému je neobratnost při zakládání složek, složky se musí vytvářet dopředu a nelze je vytvořit při zápisu souboru. Podstatnější nevýhodou je nutnost serializovat obsah souboru do textu, a nejedná se tedy o příliš vhodnou metodu ukládání např. obrazových dat kvůli nutnosti převést binární data na textový formát např. Base64. Alternativní řešení pro Expo neexistuje. Ukázka práce se souborovým systémem je v následující ukázce kódu.

\begin{lstlisting}[language=JavaScript, caption=Ukázka entity]
import * as FileSystem from 'expo-file-system';

let string = await FileSystem.readAsStringAsync(path);
\end{lstlisting}

Do proměnné string se asynchronně zapíše obsah souboru v cestě, v případě chyby čtení, např. způsobené neexistujícím souborem, se vyhodí výjimka. Soubor lze dále zpracovat předevím např. textu na JSON nebo na XML, případně jakýkoliv jiný formát, který aplikace využívá. V aplikaci se nad těmito funkcemi staví třída PersistenStorage, která poskytuje obecné metody pro uložení a získání kolekcí či jednotlivých entit.

\section{Implementace obrazovek}

Implementace obrazovek byla nejdelší částí tvorby aplikace a nejnáročnější na testování. Obrazovky jsou React komponenty umístěny ihned pod navigátorem. V jejich životním cyklu se vytváří subkomponenty, získávají data z úložišť a pomocí událostí komunikují s úložišti. Obrazovky jsou spolu se subkomponentami zodpovědné za správu interakcí s uživatelem.

Níže je uvedena ukázková implementace obrazovky AuthLoading, která rozhoduje o přesměrování uživatele po zapnutí aplikace, pokud je přihlášen či ne.

\begin{lstlisting}[language=JavaScript, caption=Ukázka implementace obrazovky]
import React from 'react';
import { StyleSheet, Text, View } from 'react-native';
import { MaterialIndicator } from 'react-native-indicators';

import AuthStore from '../store/AuthStore';
import Theme from "../constants/Theme.js";

export default class AuthLoading extends React.Component {
	verifyAuth = async () => {
		await AuthStore.awaitAuth();
		if (AuthStore.isAuthorized) {
			this.props.navigation.navigate("Map");
		} else {
			this.props.navigation.navigate("AuthContainer");
		}
	}
	
	componentDidMount() {
		this.verifyAuth();
	}
	
	render () {
		return (
		<View style={styles.container}>
			<View>
				<MaterialIndicator color={ Theme.colors.brand.primary }/>
			</View>
		</View>
		);
	}
}

const styles = StyleSheet.create({
	container: {
		flex: 1,
		backgroundColor: Theme.colors.default.background,
		alignItems: 'center',
		justifyContent: 'center',
		alignSelf: 'stretch',
	}
});

\end{lstlisting}

V ukázce lze vidět několik podstatných částí. V první části jsou importy závislostí, kde veškeré obrazovky potřebují naimportovat alespoň závislost \emph{React} z knihovny React. Na druhém řádku lze vidět import typických ovládacích prvků a utilit pro obrazovku či subkomponentu. StyleSheet je způsob zapsání stylu pro ReactNative, ukázka způsobu zápisu stylů se nachází v objektu \emph{styles} na konci ukázky kódu. Text je komponenta, která umožňuje zobrazení stylovaného textu a jako jediná může obsahovat čistý text. Komponenta View rámcově odpovídá funkci tagu \emph{div} v HTML a je pouze prázdným kontejnerem pro ostatní komponenty, ale může být stylována. Následně je vidět import \emph{MaterialIndicator} z knihovny \emph{react-native-indicators}, externí komponenty, která zobrazuje načítací spinner, aby UI nevypadala neresponzivně. Následně je importováno úložiště pro autentifikaci a seznam konstant obsahující styl aplikace.

Samotná obrazovka začíná na řádku 8, kde se nachází definice třídy komponenty. Na řádku devět se nachází metoda komponenty, ve které se volá úložiště Auth pro ověření, zda je uživatel autorizovaný, a případně se přenaviguje na správnou obrazovku. Vysoce podstatná je metoda \emph{componentDidMount} na řádku 18, která se spustí po načtení komponenty a volá již zmiňovanou funkci na ověření přihlášení. Metoda render vrací JSX obrazovky, tedy strukturu všech subkomponent. Na konci je již zmiňovaný objekt styles, obsahující styly.

Tato obrazovka je reprezentativní pro všechny obrazovky, s výjimkou hlavní mapy.

\subsection{Implementace hlavní mapy}

Pro hlavní část této obrazovky, tedy mapy, byla využita knihovna \emph{react-native-maps} z důvodu absence alternativ pro Expo. Knihovna však splňuje veškeré požadavky aplikace. Do mapy lze např. vkládat objekty typu trasa, bod i tyto body následně stylovat. Objekty do mapy se přidávají jako individuální komponenty, např. bod s tzv. infoboxem (bublinou nad bodem) se vytvoří následujícím způsobem.

\begin{lstlisting}[language=JavaScript, caption=Ukázka implementace obrazovky]
<MapView>
	<Marker
		coordinate={ { latitude: point.lat, longitude: point.lng } }
		icon={image}
		image={image}
		zIndex={zIndex}
	>
		<Callout>
			<MarkerPosition tracking={track.tracking} id={point.id}/>
		</Callout>
	</Marker>
</MapView>
\end{lstlisting}

Ukázka ilustruje způsob umístění v mapě (parametr coordinate), způsob vybrání ikony pro obě platformy (parametr icon a image) a nastavení zIndexu, který určuje pořadí vykreslování komponent vzájemně se překrývajících. Komponenta Callout obsahuje již zmiňované infowindow, které obsahuje subkomponentu MarkerPosition.

Pro implementaci vyjížděcího menu pro filtry byla využita komponenta \emph{rn-sliding-up-panel}, která plně splina zadání z kapitoly návrh. Detaily implementace lze zobrazit v příloze, jelikož jsou příliš dlouhé pro ukázky v textu této práce.

\begin{figure}[H]
	\begin{center}
		\includegraphics[width=70mm]{img/example_map.jpg}
	\end{center}
	\caption[Výsledek implementace hlavní mapy]{Výsledek implementace hlavní mapy -- zdroj: autor}
\end{figure}

Výsledek implementace mapy je srovnatelný s drátěným modelem představeným v kapitole Návrh.

\section{Implementace subkomponent}

Subkomponenty byly vytvořeny pro splnění mapové koncepce, tedy pro jednotlivé overlay funkce definované v kapitole návrh. Zbylé subkomponenty byly vytvářeny dle potřeby pro zpřehlednění aplikace, tedy ad hoc. Za zvláštní zmínku stojí implementace off-line map, která byla vytvořena čistě pro tuto aplikaci.

\subsection{Implementace off-line map}

Pro implementace off-line map bylo nutno vytvořit dvě komponenty. První komponentou jsou off-line mapy samotné, druhou komponentou způsob, jak vybrat rozsah, který má být stažen.

Off-line schopnost map je umožněna díky následujícím klíčovým možnostem: možnost načítat tzv. tile mapy (mapy uložené ve formátu obrázků s jednoznačným indexem v dimenzích x, y a z) a URI, které využívá Expo i pro ukládání souborů. Pro mapy je tedy jedno, zda načítají jednotlivé mapové čtverečky z disku, či přímo ze vzdáleného mapového serveru.

\begin{lstlisting}[language=JavaScript, caption=Off-line mapy]
import { UrlTile } from 'react-native-maps';
<MapView..>
	<UrlTile urlTemplate="..."/>
</MapView>
\end{lstlisting}

Do urlTemplate se místo vzdáleného zdroje doplní část kódu, která obsahuje parametry pro jednotlivé dimenze mapy ve formátu \emph{\{z\}/\{x\}/\{y\}.png}. Každý mapový čtvereček je následně načítán z disku. Toto chování je vhodné zapnout pouze v případě že uživatel není připojený k síti, či na přímé vyžádání uživatele.

Kvůli nedostatku času nebylo možné vytvořit komfortní uživatelský nástroj pro výběr off-line map, a vznikl proto pouze rudimentární. Z kontextového menu uživatel zapne dialog zadávání bodů, čímž se uloží příznak do aplikace, že uživatel vybírá off-line region. Uživatel je následně instruován, aby klikal do mapy.

\section{Implementace navigátorů}

Pro implementaci navigátorů byla vybrána již zmiňovaná standardní knihovna\linebreak \emph{react-navigation}, pro instalaci je využitý balíčkovací nástroj NPM.

\begin{lstlisting}[language=Bash, caption=Instalace react-navigation]
npm install react-navigation
npm install react-navigation-stack
\end{lstlisting}

Po instalaci je nejprve ve vybrané React komponentě nejprve naimportovat dané knihovny. 

\begin{lstlisting}[language=JavaScript, caption=Import knihoven pro hlavní navigátor]
import { createSwitchNavigator, createAppContainer, NavigationContainerComponent, NavigationActions } from 'react-navigation';
import { createStackNavigator } from 'react-navigation-stack';
\end{lstlisting}

Importováním těchto závislostí půjde vytvořit switch navigátor, hlavní navigační kontejner aplikace a stack navigátor. Následně je nutné naimportovat komponenty obrazovek následujícím způsobem.

\begin{lstlisting}[language=JavaScript, caption=Import obrazovek pro hlavní navigátor]
import Welcome from './src/screens/auth/Welcome'; // naimportuje komponentu obrazovky Welcome
import Login from './src/screens/auth/Login';
import Register from './src/screens/auth/Register';
import MapScreen from './src/screens/in/Map';
\end{lstlisting}

Navigátory lze následně sestavit modulárním způsobem vkládání do sebe, kde listy grafu navigátorů jsou jednotlivé komponenty obrazovek.

\begin{lstlisting}[language=JavaScript, caption=Implementace navigátorů]
const AuthContainer = createStackNavigator({
	Welcome: Welcome,
	Login: Login,
	Register: Register
}, {
	headerMode: 'none'
});

const AppNavigator = createSwitchNavigator({
	AuthLoading: AuthLoading,
	Map: MapScreen,
	AuthContainer: AuthContainer
}, {
	"initialRouteName": "AuthLoading" // nazev vychozi cesty, resp. obrazovkove komponenty
});
\end{lstlisting}

Navigátory se následně obalí aplikačním kontejnerem, ze kterého se vytvoří hlavní komponenta aplikace, tedy její vstupní bod.

\begin{lstlisting}[language=JavaScript, caption=Vytvoření aplikačního kontejneru]
const AppContainer = createAppContainer(
	AppNavigator
);

export default class App extends React.Component
{
	render () {
		return (
				<React.Fragment>
					<AppContainer ref = { setNavigatorRef }/>
					<FlashMessage position="top" />
				</React.Fragment>
		)
	}
}
\end{lstlisting}

V ukázce kódu se vytváří komponenta App, která je vstupním bodem aplikace. V komponentě App se nachází fragment, který je kolekcí různých React komponent a sám nemá v uživatelském rozhraní žádnou funkci. Následně se vytvořený AppContainer využije přímo v komponentě a jeho vykresleným obsahem se stávají různé obrazovky specifikované v předchozích ukázkách kódu. Ačkoliv se nejedná přímo o funkce navigátoru, na stejné úrovni se také nachází komponenta FlashMessage, která zajistí, že se tzv. flash zprávy vygenerované v aplikaci zobrazují napříč všemi obrazovkami z jednoho místa (tedy bez duplikace kódu). Nejedná se tedy o funkci navigátoru, ale pouze o komponentu, kterou je vhodné umístit na nejvyšší úroveň kvůli prevenci duplikaci kódu a jedná se o vhodnou ukázku, co lze na stejnou úroveň jako kontejner navigátorů vložit.

Pro přenavigování lze využít v jakékoliv komponentě umístěné pod AppContainerem (resp. pod jakýmkoliv navigátorem) funkcí uvedenou níže.

\begin{lstlisting}[language=JavaScript, caption=Přenavigování]
this.props.navigation.navigate("Map");
\end{lstlisting}

Všem komponentám je předán odkaz na navigátor a pomocí klíče (názvu) obrazovky se na ni lze přenavigovat kdekoliv uvnitř komponenty, např. při reagování na kliknutí z tlačítka či doběhnutím vnitřní události. Navigace mimo komponentu ale tímto způsobem není možná, proto byl autorem aplikace v App.tsx vytvořena možnost, jak tuto funkci z komponenty získat. V komponentě AppContainer byl specifikován atribut ref. Atribut ref v Reactu slouží k získání instance určité komponenty. Komponenta se předá funkci setNavigatorRef, který do lokální proměnné vloží instanci navigátoru. Aplikace nad touto lokální instancí vytváří funkci navigate, ve které volá událost navigace a je tedy možné z kódu mimo komponenty volat funkce navigace.

\begin{lstlisting}[language=JavaScript, caption=Přenavigování]
let instanceRef: NavigationContainerComponent;

function setNavigatorRef(instance: NavigationContainerComponent) {
	instanceRef = instance;
}

function navigate(routeName, params) {
	instanceRef.dispatch(
		NavigationActions.navigate({
			routeName,
			params,
		})
	);
}
\end{lstlisting}

\section{Implementace push notifikací}

Jednou z mnoha výhod platformy Expo je jednoduchost napojení aplikačních push notifikací. Není potřeba využívat různé poskytovatele pro různé platformy, Expo zajistí doručení na jakékoliv zařízení, které se k notifikacím registruje. Výhodou také je jednoduchost celého systému, kde stačí aby se zařízení pouze oznámilo serveru svým identifikátorem a následně lze notifikace volat za pomocí REST API Expo. Celý proces je velmi jednoduchý na implementaci v mobilní aplikaci i ve zdroji notifikací. Získání notifikací na straně aplikace lze docílit následujícím způsobem.

\begin{lstlisting}[language=JavaScript, caption=Expo notifikace]
import { Notifications } from 'expo';
import * as Permissions from 'expo-permissions';
import Constants from 'expo-constants';


const { status } = await Permissions.askAsync(Permissions.NOTIFICATIONS);

if (finalStatus !== 'granted') {
	return;
}

const token = await Notifications.getExpoPushTokenAsync();
\end{lstlisting}

Pro notifikace je třeba získat svolení uživatele. Pokud svolení dá, Expo vrátí textový token, který se následně dá použít k poslání notifikace na konkrétní zařízení. Pro zjednodušení práce s větším počtem stejných notifikací je možnost vytvářet skupiny příjemců, aplikace ji nevyužívá. Kód pro generování zpráv na straně serveru není součástí této práce a je pouze obsažen ve webovém backendu aplikace Anitra.

% \section{Sestavení pomocí TurtleCLI}

% \section{Publikování na Google Play}
\chapter{Nasazení a testování}

Tato kapitola se zabývá vytvořením aplikačního artefaktu, publikování tohoto artefaktu na distribuční platformy a následným otestováním aplikace na cílových zařízeních.

\section{Vytvoření artefaktu}

Pro vytvoření spustitelných a publikovatelných artefaktů byl vybrán již zmiňovaný způsob přes TurtleCLI. Tímto se docílí rychlého sestavení na lokálním stroji a aplikaci je možno téměř ihned nahrát do distribučních platforem Google Play a App Store.

TurtleCLI samotné lze nainstalovat přes NPM.

\begin{lstlisting}[language=Bash, caption=Instalace TurtleCLI]
npm install -g turtle-cli
\end{lstlisting}

Vytvoření artefaktu pro Android lze docílit následujícím způsobem:

\begin{lstlisting}[language=Bash, caption=Sestavení na Android,label={androidbuild}]
export EXPO_ANDROID_KEY_PASSWORD=""
export EXPO_ANDROID_KEYSTORE_PASSWORD=""
turtle build:android \
--keystore-path keystore.jks \
--keystore-alias alias -u username -p password
\end{lstlisting}

Vývojář pravděpodobně soubory pro kryptografický podpis aplikace v této fázi nemá. Stačí tedy pouze zavolat následující ukázku kódu.

\begin{lstlisting}[language=Bash, caption=Vygenerování klíčů pro Android]
expo build:android
\end{lstlisting}

Turtle automaticky uživateli vygeneruje soubory pro kryptografický podpis aplikace, nebo umožní vývojáři dodat vlastní. Po vygenerování klíčů je podstatné si soubory i heslo zálohovat na bezpečné místo, jelikož by bez těchto údajů nešlo aplikaci aktualizovat. Po vygenerování a zálohování těchto údajů lze spustit ukázku kódu \ref{androidbuild} a podepsaný soubor ve formátu \emph{.aab} se uloží do výchozí složky pro sestavené Expo aplikace, nebo do složky, kterou uživatel specifikuje.

Sestavení artefaktu pro iOS je poněkud komplikovanější záležitostí. Zprvu je nutné napsat, že sestavení pro iOS je možné jen na zařízeních s operačním systémem MacOS. Pro instalaci je taktéž potřeba více softwarových nástrojů. Zprvopočátku je potřeba nainstalovat Xcode, prostředí pro vývoj iOS aplikací, a CLI nástroje pro Xcode. Pro instalaci se také předpokládá, že bude nainstalovaný balíčkový manažer Homebrew, kterým se nainstaluje ekosystém Fastlane. Fastlane je sada CLI nástrojů, která zjednodušuje proces vydání aplikací pro iOS i Android. Pro sestavení Android aplikací není povinná, ale iOS aplikace bez tohoto nástroje TurtleCLI sestavit nedovolí.

\begin{lstlisting}[language=Bash, caption=Sestavení na iOS včetně vygenerování klíčů]
xcode-select --install #nainstaluje CLI nastroje Xcode 
brew install fastlane #nainstaluje Fastlane
fastlane init
expo build:ios #vygeneruje klice
expo fetch:ios:certs # vyexpoertuje klice 
turtle build:ios
\end{lstlisting}

Po spuštění \emph{expo build:ios} Expo CLI umožní vygenerovat distribuční certifikát, notifikační certifikát a profil fondu zařízení, které svazují testovací zařízení s autorizovanými vývojáři. Aplikaci se také vytvoří bundle ID, které slouží pro její unikátní identifikaci. Podstatné je si tyto certifikáty opět zálohovat, stejně jak vygenerované heslo k P12 distribučnímu certifikátu. Expo tyto soubory vytvoří v lokální složce a je potřeba zajistit, že se omylem nenahrají do verzovacího systému.

Při prvním spuštění se citelně dlouho stahují závislosti Expo pro sestavení aplikace a následující sestavení jsou rychlá.

\section{Publikování}

Publikování aplikace bylo zkomplikováno globální situací spojenou s COVID-19. Interní testovací verze nebylo možné publikovat ani po třech týdnech a aplikace nemohla být distribuovaná testovacím uživatelům.

\begin{figure}[H]
	\begin{center}
		\includegraphics[width=140mm]{img/covid19.png}
	\end{center}
	\caption[Varování před publikováním aplikace na Google Play]{Varování před publikováním aplikace na Google Play -- zdroj: Google Inc.}
\end{figure}

Autor aplikaci tedy otestoval sám vytvoření spustitelného souboru APK pro platformu Android a interním testováním přes TestFlight na zařízeních s iOS.

Pro testování na platformě Android byla využita schopnost systému spouštět soubory z jiných zdrojů, než oficiálních distribučních kanálů. Autor práce tedy vložil tyto soubory na disky zařízení a aplikaci následně spustil.

Pro testování na platformě iOS bylo nutno využít TestFlight z absence jiných možností řešení nasazení aplikace. Artefakty sestavení ve formátu .ipa nejdou samy o sobě přímo nahrát do obchodu App Store a oficiální i komunitní dokumentace je ve způsobu řešení těchto problémů značně neaktuální. Pro nasazení je nutné využít aplikaci Transporter dostupnost z App Store na platformě macOS. Artefakt se napáruje na správný záznam v App Store pomocí bundle ID, které bylo vytvořeno nástrojem Expo. Nahraný artefakt se následně zpracovává a v rámci minut bude dostupný pro nasazení pomocí TestFlight. Po zpracování je možné nahranou verzi aplikace testovat po dobu 90 dní. Pro získání aplikace k otestování je nutné vygenerovat si pozvánku nebo kód pro betatest.

\section{Testování}
Pro testování byla využita následující zařízení:
\begin{itemize}
	\item iPhone XS
	\item iPad
	\item Android tablet nVidia Shield.
\end{itemize}

Zařízení byla vybrána podle dostupnosti a všechna jsou fyzická. Vybrané přístroje dobře reprezentují rozsah používaných zařízení potenciálními uživateli i uživateli mobilních aplikací obecně. Od testování na reprezentativním telefonu se systémem Android muselo být odstoupeno kvůli nedostatku času.

Na každém zařízení byla aplikace nainstalována a byl proveden následující zjednodušený testovací scénář:

\begin{itemize}
	\item otevření aplikace,
	\item přihlášení přes stejný testovací účet,
	\item zobrazení posledních pozic v mapě,
	\item zapnutí aktuální pozice v mapě,
	\item zobrazení detailu trackeru,
	\item zobrazení fotky z fotogalerie,
	\item načtení trasy trackeru,
	\item načtení detailu bodu v trase,
	\item odnačtení trasy trackeru.
\end{itemize}

Cílem tohoto scénáře bylo prověřit hlavní funkce aplikace v logické návaznosti. Výsledek testovacího scénáře je uveden v následující tabulce. Za neúspěch je považována nemožnost dokončení testovacího scénáře, způsobená například pádem aplikace, nedokončeným načítáním, či jakoukoliv jinou situací, která znemožní dokončení akce.

\begin{table}[H]
	\begin{tabularx}{\textwidth}{|X|X|X|X|}
		\hline
		Zařízení        & OS      & Verze  & Výsledky testů \\ \hline
		nVidia Shield   & Android & 5.1.1  & úspěšné \\ \hline
		iPad            & iOS     & 12.3.1       & úspěšné    \\ \hline
		iPhone XS       & iOS     & 13.3.1 &  úspěšné  \\ \hline
	\end{tabularx}
	\caption[Výsledky testování aplikace]{Výsledky testování aplikace -- zdroj: autor}
\end{table}

Z výše uvedené tabulky lze usoudit, že testy proběhly velmi úspěšně a bez chyb. Aplikaci je tedy možno co nejdříve uvolnit určitému kruhu uživatelů k internímu testování a verifikovat, zda aplikace splňuje požadavky stanovené v analýze požadavků i dle dojmu uživatelů.
\chapter*{Závěr}
\addcontentsline{toc}{chapter}{Závěr}

Cílem této práce bylo vytvořit offline-capable mobilní aplikaci splňující požadavky ornitologů pro práci v poli. Tento cíl byl splněn vytvořením aplikace ve frameworku React Native, resp. Expo. Aplikace byla vytvořena a otestována autorem na obou hlavních mobilních platformách a zařízeních různých typů a splňuje téměř všechny definované funkční i nefunkční požadavky a všechny z kritických funkčních požadavků. Kvůli komplikacím způsobeným virem COVID-19 nemohla být aplikace včas nasazena. Distribuční platformy nestíhaly zpracovávat požadavky nových aplikací, a proto nebylo ani možné distribuovat testovací verzi vybrané sekci potenciálních uživatelů. 

Vývoj bude pokračovat a aplikace se stane užitečnou pomůckou při výzkumných projektech klientů platformy Anitra. Ostrý provoz aplikace se očekává po červnu 2020, kdy se do aplikace doimplementují některé chybějící moduly a aplikace se výrazně zoptimalizuje. Autor práce taktéž plánuje nahradit mapovou knihovnu knihovnou od společnosti Mapbox, která je výrazně rychlejší a nabízí více funkcí, které se do budoucna pro mobilní aplikaci plánují. Klientům bude dodána přes internetové obchody mobilních aplikací Android Google Play a iOS App Store. O vydání aplikace se dozví z newsletteru platformy Anitra či z webové stránky. Do aplikace je nutno doimplementovat zadávání bodů zájmu a zlepšit zobrazování a nahrávání tras tak, aby fungovalo na pozadí aplikace a minimalizovalo spotřebu energie.

Tato práce se věnovala vývoji mobilní aplikace pro podporu práce ornitologů v poli. Autor zmapoval historický vývoj ornitologie, telemetrie zvířat a úskalí vědecké práce, jejichž pochopení je klíčovým předpokladem vývoje úspěšné aplikace. V následující kapitole byla rozebrána existující řešení různých problémů, se kterými se ornitologové potýkají při práci v terénu i v přípravě k těmto činnostem. Další kapitoly se věnovaly běžnému postupu při návrhu a implementaci malých softwarových aplikací. Nejpre byli identifikováni stakeholdeři, sesbírány požadavky a z nich byly sestaveny funkční i nefunkční požadavky na aplikaci. V následující kapitole byla vybrána technologie pro řešení aplikace. Kapitola Návrh se věnovala návrhu uživatelského rozhraní i softwarové architektury aplikace, v kapitole Implementace se tyto navrhnuté modely realizovaly. V poslední kapitole Nasazení a testování byla aplikace testována na vhodných zařízeních a v testu obstála.

%%% Seznam použité literatury
\include{literatura}

%%% Přílohy k bakalářské práci, existují-li. Každá příloha musí být alespoň jednou
%%% odkazována z vlastního textu práce. Přílohy se číslují.
\part*{Přílohy}
\appendix
\chapter{Zdrojové kódy aplikace}

Zdrojové kódy k aplikaci i text této práce je dostupný z \url{https://github.com/luxor-git/birdstracker-app}. Zdrojový kód ve stavu k 11. 5. 2020 je přiložen jako příloha této práce.
% \include{...}

\end{document}
