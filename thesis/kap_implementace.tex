%%% Fiktivní kapitola s ukázkami sazby

\chapter{Implementace}

Kapitola implementace se věnuje konkrétním akcím vedoucím k vytvoření spustitelného aplikačního artefaktu. V jednotlivých podkapitolách budou uvedeny využité prostředky, konkrétní kroky a jejich výstupy. Některé z těchto kroků jsou použitelné pro všechny Expo projekty a některé úsudky lze využit při vytváření vlastních Expo projektů. Za zmínku zde také stojí kapitola Implementace off-line map, ve které je popsán způsob jak zajistit funkční off-line mapy pro Expo projekty, pro co neexistují zatím žádné komunitní knihovny nebo Expo-nativní způsoby řešení.

\section{Vývojové prostředí}

% https://www.techopedia.com/definition/16376/development-environment

Vývojové prostředí je kolekce procedur a nástroj pro vývoj, testování a ladění aplikací nebo programů (ref). Termín vývojové prostředí se používá v synonymu s termínem IDE, integrovaným vývojový prostředím, což je softwarový nástroj pro psaní, stestavení, testování a ladění programů. Do tohoto termínu taktéž lze zahrnout i nástroje co IDE samotné využívá pro setavení artefaktů.

Pro vývoj aplikace byly zvoleny následující nástroje vývojového prostředí:

\begin{itemize}
	\item VisualStudio Code jako nástroj IDE,
	\item Git jako verzovací nástroj,
	\item GitHub jako nástroj verzování i projektového řízení,
	\item AndroidStudio pro správu Android virtuálních strojů,
	\item XCode pro správu iOS virtuálních strojů (pouze na MacOS).
\end{itemize}

Další klíčový softwarový nástroj je taktéž instalace Node.js a balíčkovacího nástroje NPM, bez kterých by vývoj Expo aplikce nemožný. Text této práce se jejich instalaci nevěnuje a předpokládá, že již nainstalované jsou. Instalace na běžných a aktuálních verzích OS není problémem.

Do pojmu vývojové prostředí se zahrnují i izolovaná prostředí, ve kterém běží testovací nebo produkční instance aplikace (resp. datového zdroje), v tomto případě platformy Anitra. Pro vývoj této aplikace nebylo toto rozlišení nutné a vyvílejo se přímo na produkční verzi platformy Anitra.

\subsection{Vytvoření projektu}

V ukázce kódu \ref{expoinstall} bylo již zmíněno, jakým způsobem vytvořit Expo projekt. Pro vytvoření expo projektu stačí zadat následující příkazy do příkazové řádky.

\begin{lstlisting}[language=Bash, caption=Vytvoření nového projektu]
npm install --global expo-cli
expo init mobilni-aplikace
\end{lstlisting}

První řádek nainstaluje \emph{expo-cli}, nástroje pro vytváření, správu, spouštění a sestavení Expo projektů. Druhý příkaz vytvoří nový expo projekt \emph{mobilni-aplikace} v aktuální složce uživatele.

Z předpřipraveného kódu je dobré zminit vstupní bod aplikace \emph{App.tsx}, ve kterém je připravená ukázková komponenta. Z tohoto vstupního bodu se implementují navigátory zmíněné v kapitole Implementace navigátorů.

\section{Implementace obrazovek}

\section{Implementace komponent}

\subsection{Implementace off-line map}

Pro implementace off-line map bylo nutno vytvořit dvě komponenty. První komponentou jsou off-line mapy samotné, druhou komponentou způsob, jak vybrat rozsah, který má být stáhnut.

Off-line schopnost map je umožněna díky následujícím klíčovým možnostem: možnost načítat tzv. tile mapy (mapy uložené ve formátu obrázků s jednoznačným indexem v dimenzích x, y a z) a URI, které využívá Expo i pro ukládání souborů. Pro mapy je tedy jedno, zda načítají jednotlivé mapové čtverečky z disku, či přímo ze vzdáleného mapového serveru.

\begin{lstlisting}[language=JavaScript, caption=Off-line mapy]
import { UrlTile } from 'react-native-maps';
<MapView..>
	<UrlTile urlTemplate="..."/>
</MapView>
\end{lstlisting}

Do urlTemplate se místo vzdáleného zdroje doplní část kódu, která obsahuje parametry pro jednotlivé dimenze mapy ve formátu \emph{\{z\}/\{x\}/\{y\}.png}. Každý mapový čtvereček je následně načítán z disku. Toto chování je vhodné zapnout pouze v případě že uživatel není připojený k síti, či na přímé vyžádání uživatele.

Z časových nedostatků nebylo možno vytvořit komfortní uživatelský nástroj pro výběr off-line map.

\section{Implementace navigátorů}

Pro implementaci navigátorů byla vybrána již zmiňovaná standardní knihovna \emph{react-navigation}. Pro instalaci je využitý balíčkovací nástroj NPM.

\begin{lstlisting}[language=Bash, caption=Instalace react-navigation]
npm install react-navigation
npm install react-navigation-stack
\end{lstlisting}

Po instalaci je nejprve ve vybrané React komponentě nejprve naimportovat dané knihovny. 

\begin{lstlisting}[language=JavaScript, caption=Import knihoven pro hlavní navigátor]
import { createSwitchNavigator, createAppContainer, NavigationContainerComponent, NavigationActions } from 'react-navigation';
import { createStackNavigator } from 'react-navigation-stack';
\end{lstlisting}

Importováním těchto závislostí půjde vytvořit switch navigátor, hlavní navigační kontejner aplikace a stack navigátor. Následně je nutno naimportovat komponenty obrazovek následujícím způsobem.

\begin{lstlisting}[language=JavaScript, caption=Import obrazovek pro hlavní navigátor]
import Welcome from './src/screens/auth/Welcome'; // naimportuje komponentu obrazovky Welcome
import Login from './src/screens/auth/Login';
import Register from './src/screens/auth/Register';
import MapScreen from './src/screens/in/Map';
\end{lstlisting}

Navigátory lze následně sestavit modulárním způsobem vkládání do sebe, kde listy grafu navigátorů jsou jednotlivé komponenty obrazovek.

\begin{lstlisting}[language=JavaScript, caption=Implementace navigátorů]
const AuthContainer = createStackNavigator({
	Welcome: Welcome,
	Login: Login,
	Register: Register
}, {
	headerMode: 'none'
});

const AppNavigator = createSwitchNavigator({
	AuthLoading: AuthLoading,
	Map: MapScreen,
	AuthContainer: AuthContainer
}, {
	"initialRouteName": "AuthLoading" // nazev vychozi cesty, resp. obrazovkove komponenty
});
\end{lstlisting}

Navigátory se následně obalí aplikačním kontejnerem, ze kterého se vytvoří hlavní komponenta aplikace, tedy její vstupní bod.

\begin{lstlisting}[language=JavaScript, caption=Vytvoření aplikačního kontejneru]
const AppContainer = createAppContainer(
	AppNavigator
);

export default class App extends React.Component
{
	render () {
		return (
				<React.Fragment>
					<AppContainer ref = { setNavigatorRef }/>
					<FlashMessage position="top" />
				</React.Fragment>
		)
	}
}
\end{lstlisting}

V ukázce kódu se vytváří komponenta App, která je vstupním bodem aplikace. V komponentě App se nachází fragment, který je kolekcí různých React komponent a sám nemá v uživatelském rozhraní žádnou funkci. Následně se vytvořený AppContainer využije přímo v komponentě a jeho vykresleným obsahem se stávají různé obrazovky specifikované v předchozích ukázkách kódu. Ačkoliv se nejedná přímo o funkce navigátoru, na stejné úrovni se také nachází komponenta FlashMessage, která zajistí, že se tzv. flash zprávy vygenerované v aplikaci zobrazují napříč všemi obrazovkami z jednoho místa (tedy bez duplikace kódu). Nejedná se tedy o funkci navigátoru, ale pouze o komponentu, kterou je vhodno umístit na nejvyšší úroveň kvůli prevenci duplikaci kódu a jedná se o vhodnou ukázku, co lze na stejnou úroveň jako kontejner navigátorů vložit.

Pro přenavigování lze využít v jakékoliv komponentě umístěné pod AppContainerem (resp. pod jakýmkoliv navigátorem) funkcí uvedenou níže.

\begin{lstlisting}[language=JavaScript, caption=Přenavigování]
this.props.navigation.navigate("Map");
\end{lstlisting}

Všem komponentám je předán odkaz na navigátor a pomocí klíče (názvu) obrazovky se na ni lze přenavigovat kdekoliv uvnitř komponenty, např. při reagování na kliknutí z tlačítka či doběhnutím vnitřní události. Navigace mimo komponentu ale tímto způsobem není možná, proto byl autorem aplikace v App.tsx vytvořena možnost, jak tuto funkci z komponenty získat. V komponentě AppContainer byl specifikován atribut ref. Atribut ref v Reactu slouží k získání instance určité komponenty. Komponenta se předá funkci setNavigatorRef, který do lokální proměnné vloží instanci navigátoru. Aplikace nad touto lokální instancí vytváří funkci navigate, ve které volá událost navigace a je tedy možné z kódu mimo komponenty volat funkce navigace.

\begin{lstlisting}[language=JavaScript, caption=Přenavigování]
let instanceRef: NavigationContainerComponent;

function setNavigatorRef(instance: NavigationContainerComponent) {
	instanceRef = instance;
}

function navigate(routeName, params) {
	instanceRef.dispatch(
		NavigationActions.navigate({
			routeName,
			params,
		})
	);
}
\end{lstlisting}

\section{Implementace push notifikací}

Jednou z mnoha výhod platformy Expo je jednoduchost napojení aplikačních push notifikací. Není potřeba využívat různé poskytovatele pro různé platformy, Expo zajistí doručení na jakékoliv zařízení, které se k notifikacím registruje. Výhodou také je jednoduchost celého systému, kde stačí aby se zařízení pouze oznámilo serveru svým identifikátorem a následně lze notifikace volat za pomocí REST API Expo. Celý proces je velmi jednoduchý na implementování v mobilní aplikaci i ve zdroji notifikací. Získání notifikací na straně aplikace lze docílit následujícím způsobem.

\begin{lstlisting}[language=JavaScript, caption=Expo notifikace]
import { Notifications } from 'expo';
import * as Permissions from 'expo-permissions';
import Constants from 'expo-constants';


const { status } = await Permissions.askAsync(Permissions.NOTIFICATIONS);

if (finalStatus !== 'granted') {
	return;
}

const token = await Notifications.getExpoPushTokenAsync();
\end{lstlisting}

Pro notifikace je nutno získat uživatelovo svolení. Pokud svolení dá, Expo vrátí textový token, který se následně dá použít pro poslání notifikace na konkrétní zařízení. Pro zjednodušení práce s větším počtem stejných notifikací je možnost vytvářet skupiny příjemců, aplikace tohoto nevyužívá. Kód pro generování zpráv na straně serveru není součástí této práce a je pouze obsažen ve webovém backendu aplikace Anitra.
