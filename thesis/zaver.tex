\chapter*{Závěr}
\addcontentsline{toc}{chapter}{Závěr}

Cílem této práce bylo vytvořit offline-capable mobilní aplikaci splňující požadavky ornitologů pro práci v poli. Tento cíl byl splněn vytvořením aplikace ve frameworku React Native, resp. Expo. Aplikace byla vytvořena a otestována autorem na obou hlavních mobilních platformách a zařízeních různých typů a splňuje téměř všechny definované funkční i nefunkční požadavky a všechny z kritických funkčních požadavků. Kvůli komplikacím způsobeným virem COVID-19 nemohla být aplikace včas nasazena. Distribuční platformy nestíhaly zpracovávat požadavky nových aplikací, a proto nebylo ani možné distribuovat testovací verzi vybrané sekci potenciálních uživatelů. 

Vývoj bude pokračovat a aplikace se stane užitečnou pomůckou při výzkumných projektech klientů platformy Anitra. Ostrý provoz aplikace se očekává po červnu 2020, kdy se do aplikace doimplementují některé chybějící moduly a aplikace se výrazně zoptimalizuje. Autor práce taktéž plánuje nahradit mapovou knihovnu knihovnou od společnosti Mapbox, která je výrazně rychlejší a nabízí více funkcí, které se do budoucna pro mobilní aplikaci plánují. Klientům bude dodána přes internetové obchody mobilních aplikací Android Google Play a iOS App Store. O vydání aplikace se dozví z newsletteru platformy Anitra či z webové stránky. Do aplikace je nutno doimplementovat zadávání bodů zájmu a zlepšit zobrazování a nahrávání tras tak, aby fungovalo na pozadí aplikace a minimalizovalo spotřebu energie.

Tato práce se věnovala vývoji mobilní aplikace pro podporu práce ornitologů v poli. Autor zmapoval historický vývoj ornitologie, telemetrie zvířat a úskalí vědecké práce, jejichž pochopení je klíčovým předpokladem vývoje úspěšné aplikace. V následující kapitole byla rozebrána existující řešení různých problémů, se kterými se ornitologové potýkají při práci v terénu i v přípravě k těmto činnostem. Další kapitoly se věnovaly běžnému postupu při návrhu a implementaci malých softwarových aplikací. Nejpre byli identifikováni stakeholdeři, sesbírány požadavky a z nich byly sestaveny funkční i nefunkční požadavky na aplikaci. V následující kapitole byla vybrána technologie pro řešení aplikace. Kapitola Návrh se věnovala návrhu uživatelského rozhraní i softwarové architektury aplikace, v kapitole Implementace se tyto navrhnuté modely realizovaly. V poslední kapitole Nasazení a testování byla aplikace testována na vhodných zařízeních a v testu obstála.